\documentclass{article}
\usepackage{ctex}

\title{拉普拉斯变换}
% \author{张汉杰}
\author{Vainjoker}
\date{\today}

\setCJKmainfont{WenQuanYiZenHei}

\begin{document}
    \maketitle
    \newpage

    \section{四种基本信号}

    因果信号$e^{\alpha t}\epsilon(t)$
    反因果信号$e^{\beta t}\epsilon(-t)$
    双边信号$e^{\alpha t}\epsilon(t)+e^{\beta t}\epsilon(-t)$
    
    \section{线性的判定}
    时域:描述系统的违法方程域$y_f(t)$均为实项的形式
    
    频域:比较$f(t)y_f(t)$中没有新的频域分量组成

    \section{信号取样}

    模拟信号数字化的第一步(取样,量化,编码

    信号的时间离散
    冲激取样$frac{1}{T}\sigma_{-\infty}{\infty}F(\omega-nfrac{2\pi}{T})$

    若f(t) 为频带受限信号  非周期 连续

    从f(t) 取样信号中实现  离散 周期

    无失真性质的条件  连续 非周期


    


\end{document}
