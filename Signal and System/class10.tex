\documentclass{article}
\usepackage{ctex}

\title{信号的正交函数集分率}
% \author{张汉杰}
\author{Vainjoker}
\date{\today}

\setCJKmainfont{WenQuanYiZenHei}

\begin{document}
    \maketitle
    \newpage

    \section{正交}

    $\int_{t_1}^{t_2}\varphi_1(t)\varphi_2(t)=0$
    $\varphi_1(t),\varphi_2(t)$在[t1,t2]正交

    \subsection{正交函数集}

\subsection{完备的正交函数集}

对于给定的正交函数集,不能找到之外的函数$\phi(t)$满足正交关系

\subsection{三角函数集}

[0,T]所有余弦,所有正弦 函数


$\int_{0}^{T}\cos(m\omega t)\cos(n\omega t)dt=frac{1}{2}\int_{0}^{T}[\cos(m+n)\omega t+\cos(m-n)\omega t]dt=\frac{1}{2}frac{sin(m+n)\omega t}{(m+n)\omega}$

\subsection{三角傅里叶级数}

以三角函数对周期信号进行分解-->三角傅里叶级数

频率代表单位时间内信号波形的变换次数

$a_n=frac{2}{T}\int_{-frac{T}{2}}^{frac{T}{2}}f(t)\cos\omega tdt$

$b_n=frac{2}{T}\int_{-frac{T}{2}}^{frac{T}{2}}f(t)\sin\omega tdt$

$f(t)$为偶函数$b_n=0$

$f(t)$为奇函数$a_n=0$

$f(t)$为偶谐函数f(t)前半周期和后半周期波形相同

$f(t)$为奇谐函数f(t)前半周期和后半周期波形关于横轴对称

\subsection{信号的频域构成}
频谱

振幅相位



   

\end{document}
