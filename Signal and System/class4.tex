\documentclass{article}
\usepackage{ctex}

\title{<++>}
% \author{张汉杰}
\author{Vainjoker}
\date{\today}

\setCJKmainfont{WenQuanYiZenHei}

\begin{document}
    \maketitle
    \newpage
    
    \section{系统的初始值}%

        初始值:是n阶系统在t=0时接入激励,其响应在t=0时刻的值

        初始状态:是系统在激励尚未接触t=0时刻的响应值,反映系统的历史情况
        微分方程等号右端有$\delta(t)$时仅在等号左端$y(t)$的最高阶导数中含有$\delta(t)$

        
        由系统的概念

        获得系统的描述方法(框图,方程,框图和方程之间的相互转化)

        根据方程求解系统(齐次解+特解)

        为求解方程需要初始值(通过系数匹配)解出初始值

    \section{零输入响应}%
    $y_{x}(t)=y_{zi}(t)$由状态$x$产生的
    
    \subsection{初始值的确定}%

    $y(0_-)=y_{zi}(0_-)=y_{x}(0_-)$

    $y(0_+)=y_{zi}(0_+)+y_{zs}(0_+)$

    零输入响应对应齐次微分方程,故不存在跃变

    即: 

    $y_{zi}(0_+)=y_{zi}(0_-)+=y(0_-)$
    
    \subsection{求解步骤}%
        \subsubsection{设定齐次解}%
        \subsubsection{代入初始值,求待定系数}%
    
    \section{零状态响应}% 
    $y_{f}(t)=y_{zs}(t)$由输入$f$产生的

    \subsection{初始值的确定}%

    $y(0_-)=y_{zs}(0_-)=0$

    由系数匹配法,从$y_{zs}(0_-)=0$求$y_{zs}(0_+)$

    先求$y_{zi}(0_+)$再求$y_{zs}(0_+)=y_(0_+)-y_{zi}(0_+)$

    \subsection{求解步骤}%
        \subsubsection{设定齐次解}%
        \subsubsection{设定特解,代入方程求解}%
        \subsubsection{代入初始值,求待定系数}%

    \section{全响应}%

    $y(t)=y_x(t)+y_f(t)$
    

\end{document}
