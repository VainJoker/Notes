\documentclass{article}
\usepackage{ctex}

\title{冲激响应和阶跃响应}
% \author{张汉杰}
\author{Vainjoker}
\date{\today}

\setCJKmainfont{WenQuanYiZenHei}

\begin{document}
    \maketitle
    \newpage

    \section{冲激响应}%
    冲激响应是由单位冲激函数$\delta(t)$引起的零状态响应记为$h(t)$

    $h(t)$ 隐含条件:

    $f(t)=\delta (t)$

    $h(0_-)=h'(0_-)=0$ (对二阶系统)

    \subsection{求解方法}%
    $y''(t)+a_1y'(t)+a_0y(t)=b_2f''(t)+b_1f'(t)+b_0f(t)$

    分两步进行选取新变量 $h_1(t)$ 使他满足

    $h_1''(t)+a_1h_1'(t)+a_0h_1(t)=\delta(t)$
    采用经典法求解$h_1(t)$根据LTI系统零状态响应和线性性质和微分特性,则冲激响应:

    $h(t)=b_2h_1''(t)+b_1h_1'(t)+b_0h_1(t)$


    \section{阶跃响应}%
    阶跃响应是单位阶跃函数$\epsilon(t)$引起的零状态响应记为$g(t)$


    $g(t)$ 隐含条件:

    $f(t)=\epsilon (t)$

    $g(0_-)=g'(0_-)=0$ (对二阶系统)

    \subsection{求解方法}%

    \subsubsection{方法一}%
    
    $y''(t)+a_1y'(t)+a_0y(t)=b_2f''(t)+b_1f'(t)+b_0f(t)$
    
    分两步进行选取新变量 $g_1(t)$ 使他满足

    $g_1''(t)+a_1g_1'(t)+a_0g_1(t)=\delta(t)$
    采用经典法求解$g_1(t)$根据LTI系统零状态响应和线性性质和微分特性,则j阶跃响应:

    $g(t)=b_2g_1''(t)+b_1g_1'(t)+b_0g_1(t)$

    \subsubsection{方法二}%
    利用单位阶跃函数和单位冲激函数的关系

    $\delta(t)=\int^{t}_{-\infty}\delta(\tau)d\tau$,
    $\delta(t)=\frac{d\epsilon(t)}{dt}$

    故:
    $g(t)= \int_{-\infty}^t\delta(\tau)d\tau$,
    $h(t)=\frac{dg(t)}{dt}$


\end{document}

%%% Local Variables:
%%% mode: latex
%%% TeX-master: "class2"
%%% End:
