\documentclass{article}
\usepackage{ctex}

\title{系统分析}
% \author{张汉杰}
\author{Vainjoker}
\date{\today}

\setCJKmainfont{WenQuanYiZenHei}

\begin{document}
    \maketitle
    \newpage

    \section{连续时间系统时域分析}%:
    常系数线性微分方程时域求解
    \subsection{经典解法}%
    $y(t)=y_n(t)+y_p(t)$

    完全解=齐次解+特解

    齐次解为形如$e^{t\lambda} $的线性组合,是对应齐次微分方程的解:

    特征根为单实根齐次解$Ce^{t\lambda}$

    特征根为二重实根齐次解$(C_1t+C_0)e^{t\lambda}$

    特征根为一对共轭复根$(C\cos(\beta t)+D\sin(\beta t))e^{\alpha t}$

    特解取决于激励信号

    {未完待续}

    \subsection{奇异函数平衡法}%

    分析$y^(n)(t)$中包含的奇异函数判断$0_-$到$0_+$的连续性

    对微分方程从$0_-$到$0_+$做定积分求取$y^(n)(0_+)$

\end{document}
