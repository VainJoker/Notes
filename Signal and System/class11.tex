\documentclass{article}
\usepackage{ctex}

\title{傅里叶变换}
% \author{张汉杰}
\author{Vainjoker}
\date{\today}

\setCJKmainfont{WenQuanYiZenHei}

\begin{document}
    \maketitle
    \newpage


\section{傅里叶变换}

周期信号的频谱取极限得到的非周期信号频谱连续幅度无限小

$F(j\omega)=\int_{-\infty}^{\infty}f(t)e^{-j\omega t}dt$
   
称$F(j\omega)$为f(t)de 傅里叶变换
$F(j\omega)$一般为复函数写为$F(j\omega)=|F(j\omega)|e^{j\omega \phi}$

\section{傅里叶反变换}

$f(t)=frac{1}{2\pi}\int_{-\infty}^{\infty}F(j\omega)e^{j\omega t} d\omega$

\section{性质}

\subsection{线性}


$f_1(t)<->F_1(j/omega)$

$f_2(t)<->F_2(j/omega)$

$af_1(t)+bf_2(t)<->aF_1(j/omega)+bF_2(j/omega)$

\subsection{奇偶性}

若$f(t)$为偶$F(j/omega)$实偶

若$f(t)$为奇$F(j/omega)$虚奇

\subsection{对称性}

$f(t)<->F(j/omega)$

$f(jt)<->2\pi F(-omega)$

\subsection{尺度变换}

$f(at)<->frac{1}{|a|}F(frac{j/omega}{a})$

\subsection{时移}

$f(t+t_0)<->F(j/omega)e^{+/-j/omega t_0}$

\subsection{频移}

$f(t)e^j/omega t<->F(j/omega - \omega_0)$

\subsection{卷积性质}

$f_1(t)*f_2(t)<->F_1(j/omega)F_2(j/omega)$

$f_1(t)f_2(t)<->frac{1}{a\pi}F_1(j/omega)F_2(j/omega)$

\subsection{时域微积分}

$f'(t)=f'(t)*\delta(t)=f(t)*\delta'(t)<-->F(j/omega)(j\omega)$

$f^n(t)<-->F(j/omega)(j\omega)^n$

$f^{-1}(t)=f^{-1}(t)*\delta(t)=f(t)*\sigma(t)<-->F(j/omega)[\pi \delta(\omega)+frac{1}{(j\omega)}](j\omega)$

\section{周期信号的傅里叶变换}

$f(t)=/sum_{-\infty}^{\infty}F_ne^{j/omega t}$

$f(t)<-->2\pi /sum_{-\infty}^{\infty} F_n\delta(/omega-n/omega)$

已知$F_n$就得到指数展开式周期信号当作单个脉冲与周期性冲击序列的卷积
\end{document}
